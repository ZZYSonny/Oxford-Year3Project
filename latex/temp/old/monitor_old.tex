//TODO: Describing Java Monitor.

//I find the state graph useful for explaining the CSP implementation

Similarly, the implementation of monitor requires a server process, like the server process of mutable variable.

//TODO: Change to TikZ diagram
\begin{itemize}
    \item Before a thread enters the \textbf{synchronized} block, the thread may need to wait other process to leave the block, or more concisely, changing from \textbf{Runnable} state to \textbf{Running} state. It must first synchronize with the server process on a \textbf{WaitEnter} event.
    \item Before the process leaves the synchronized block, it must synchronize on an \textbf{Exit} event, to switch to \textbf{Finished} state.
    \item The running process may wait until some variable to change. For this, the process must send a \textbf{Wait} event to the server, indicating changing from \textbf{Running} state to \textbf{Waiting} state. Later when resuming from \textbf{Waiting} state, the process first needs to switch to \textbf{Runnable} state with a \textbf{WaitNotify} event, then a \textbf{WaitEnter} event.
    \item Also inside the block, the running process may send a \textbf{Notify} or \textbf{NotifyAll} event to notify one or all process in waiting state.
\end{itemize}

The server process should behave differently when there is a \textbf{Running} process or not. When there is no process in \textbf{Running} state, the server process should only non-deterministically pick a process waiting to enter the synchronized block. When there is a process in \textbf{Running} state, the server process should only be willing to accept 

There are a few corner cases to consider.

In Java, when a process calls notify, but there is no waiting process to wake up, the function call does nothing. However, if the server process sends a \textbf{Notify} event when no process is waiting, the server process will simply deadlock. Thus, it is essential to prevent the server process doing a \textbf{Notify} event in such case.

 
The monitor may spuriously wake up some processes at some time. For this purpose, the server process may non-deterministically choose to wake up processes whenever possible. //TODO: Livelock.

Now that waiting processes may be waked up spuriously, it becomes helpful to use \textbf{wait} with a while loop like \textbf{while(cond) wait()}, to check if variable has actually been changed. While loop, again, is not in CSP. Also \textbf{cond} is a bit unusual, because when evaluating the condition, the process may need to read global variable. For this reason, a generic while loop is defined. //TODO: While loop
